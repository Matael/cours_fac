\documentclass[a4paper,fontsize=7.5pt]{scrreprt}

% Preamble {{{1

% Packages
\usepackage[utf8]{inputenc}
\usepackage[T1]{fontenc}
\usepackage{amsmath}
\usepackage{amsfonts}
\usepackage{textcomp}
\usepackage[french]{babel}
\usepackage{enumerate}
\usepackage{pdflscape}
\usepackage{multicol}
\usepackage{vmargin}

\setmarginsrb{1.1cm}{1.0cm}% left, top
		      {1.1cm}{1.0cm}% right, down
                	      {7mm}{0.5cm}% head: height, distance
	               {7mm}{0.5cm}% foot: height, distance

% Set column space
\setlength{\columnsep}{0.25em}

% Redefine sections
\makeatletter
\renewcommand{\section}{\@startsection{section}{1}{0mm}
	{-1.7ex}{0.7ex}{\normalfont\large\bfseries}}
\renewcommand{\subsection}{\@startsection{subsection}{2}{0mm}
	{-1.7ex}{0.5ex}{\normalfont\normalsize\bfseries}}
\makeatother

% No section numbers
\setcounter{secnumdepth}{0}

% set Version
\newcommand{\Version}[0]{1.5}

% No indentation
\setlength{\parindent}{0em}

% No header and footer
\pagestyle{empty}


% Maths
\newcommand{\Nat}{\mathbb{N}}
\newcommand{\grad}{\overrightarrow{\mathrm{grad}}}
\newcommand{\rot}{\overrightarrow{\mathrm{rot}}}
% intégrale de flux :
\newcommand{\fint}{%
    \setbox0=\hbox{$\displaystyle \int\!\!\!\!\!\int$}
    \setbox1=\hbox to \wd0{\hfill$\bigcirc$\hfill}
    \setbox0=\hbox to \wd0{\copy0\hss\copy1}
    \mathop{\copy0}
}
\newcommand{\tint}{\int\!\!\!\!\!\int\!\!\!\!\!\int}
\newcommand{\dint}{\int\!\!\!\!\!\int}


%%%%%%%%%%%%%%%%%%%%%%%%%%%%%%%%%%%
%%%%%%%%%% DOCUMENT
%%%%%%%%%%%%%%%%%%%%%%%%%%%%%%%%%%%

\begin{document} % {{{1
\begin{landscape}

\begin{multicols}{3}
[\begin{center}\section{Maths}\end{center}]

\subsection{Sommes} % {{{2

$$\sum_{i=1}^N i = \frac{N(N+1)}{2}$$

$$\sum_{i=1}^N i^2 = \frac{N(N+1)(2N+1)}{6}$$

$$\sum_{k\in\Nat} \frac{\lambda^k}{k!} = e^\lambda$$

Pour $|x| < 1$ :

$$\sum_{k=1}^{+\infty} x^k = \frac{x}{1-x}$$

\subsection{Comparaisons} % {{{2

$$|x+y| \leq |x| + |y|$$

\subsection{Transformées -- Changement de domaine} %{{{2

\subsubsection{Fourier continue}

$$\mathcal{F}\{u(t)\} = U(f)  = \int_{-\infty}^{+\infty}u(t)e^{-2i\pi ft}\mathrm{d}t$$

\subsubsection{Fourier Discrète}

Avec $s[n]$ un signal discret de $N$ points :

$$\mathcal{F}\{s[n]\} = S[k] = \sum_{n=0}^{N-1} s[n]e^{-2i\pi k\frac{n}{N}}$$


\subsection{Analyse Vectorielle} % {{{2

\subsection{Laplacien (cartésien)}

$$\Delta = \nabla^2 = \frac{\partial^2}{\partial x^2} + \frac{\partial^2}{\partial y^2} +\frac{\partial^2}{\partial
z^2}$$

\subsection{D'alembertien (cartésien)}

Avec $k$ une constante :

$$\Box = \Delta - k\frac{\partial^2}{\partial t^2}$$

\subsection{Produits scalaires et vectoriels}

$$\overrightarrow{u}\cdot\overrightarrow{v} = u_xv_x+u_yv_y+u_zv_z$$

$$||\overrightarrow{u}\cdot\overrightarrow{v}|| =
||\overrightarrow{u}||||\overrightarrow{v}|||\cos(\overrightarrow{u},\overrightarrow{v})|$$

$$||\overrightarrow{u}\wedge\overrightarrow{v}|| =
||\overrightarrow{u}||||\overrightarrow{v}|||\sin(\overrightarrow{u},\overrightarrow{v})|$$

\subsection{Gradient, divergence, rotationnel}

Avec $\overrightarrow{W}$ un champ vectoriel.

$$\overrightarrow{\mathrm{grad}}f = \overrightarrow{\nabla}f$$

$$\mathrm{div}\overrightarrow{F} = \overrightarrow{\nabla}\cdot\overrightarrow{W}$$

$$\overrightarrow{\mathrm{rot}}\overrightarrow{W} = \overrightarrow{\nabla}\wedge\overrightarrow{W}$$

\subsection{Propriétés}

$$\rot\grad f = \overrightarrow{0} \Rightarrow \exists f / \overrightarrow{W} = \grad f$$

$$\grad(fg) = f\grad g + g \grad f$$

$$\mathrm{div}(f\overrightarrow{W}) = f\mathrm{div}\overrightarrow{W} + \overrightarrow{W}\cdot\grad f$$

$$\rot(f\overrightarrow{W}) = f\rot\overrightarrow{W} + (\grad f)\wedge\overrightarrow{W}$$

\subsection{Theorèmes intégraux}

$\tau$ est un volume limité par la surface fermée $\Sigma$ orientée vers l'extérieur.
$S$ est une surface appuyée sur et orientée vers le contour fermé $C$.

\subsubsection{Stokes}

$$\oint_C\overrightarrow{W}(\overrightarrow{r})\cdot\mathrm{d}\overrightarrow{r} =
\dint_S\rot\overrightarrow{W}(\overrightarrow{r})\cdot\mathrm{d}\overrightarrow{S}$$

\subsubsection{Stokes-Ostrogradski}

$$\fint_\Sigma\overrightarrow{W}(\overrightarrow{r})\cdot\mathrm{d}\overrightarrow{S} =
\tint_\tau\mathrm{div}\overrightarrow{W}(\overrightarrow{r})\cdot\mathrm{d}\tau$$


\subsection{Trigonométrie} % {{{2

$$\sin (a + b) = \sin a \cos b + \cos a \sin b$$
$$\cos (a + b) = \cos a \cos b - \sin a \sin b$$
$$e^{i\theta} = \cos\theta + i\sin\theta$$
$$\cos\theta = \frac{e^{i\theta} + e^{-i\theta}}{2}$$
$$\sin\theta = \frac{e^{i\theta} - e^{-i\theta}}{2}$$
$$e^{i\pi} + 1 = 0$$


\subsection{Suites} % {{{2

\subsubsection{Arithmétiques}

$$\begin{cases}
        u_{n_0} = a &\\
        \forall n > n_0, & u_{n+1} = u_n + r
\end{cases}$$

Terme général : 

$$u_n = a + (n-n_0)r$$

Somme des $n$ premiers termes : 

$$\sum_{p = 0}^n u_p = \frac{n+1}{2}(u_0 + u_n)$$

\subsubsection{Géométriques}

$$\begin{cases}
        u_{n_0} = a &\\
        \forall n > n_0, & u_{n+1} = qu_n
\end{cases}$$

Terme général : 
TODO
Somme des $n$ premiers termes : 

$$\sum_{p = 0}^n u_p = \frac{n+1}{2}(u_0 + u_n)$$
\end{multicols}
\end{landscape}
\end{document}
