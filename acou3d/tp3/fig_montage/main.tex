\documentclass[tikz]{standalone}
%\usetikzlibrary{...}% tikz package already loaded by 'tikz' option
\newcommand{\source}[1][]{
    \draw[thick] (0,0) rectangle (1.5,2.5);
    \draw (.75,2.5) node[above] {#1};
    \draw (.25,.8) rectangle (1.25,1);
    \draw (0,0) -- (.35,.8);
    \draw (1.5,0) -- (1.15,.8);
}
\newcommand{\onde}[1]{
    \draw[#1,thick,->,>=stealth] (0,0) -- (0,-1);
    \draw[#1] (-1.5,-.4) -- (1.5,-0.4);
    \draw[#1] (-1.5,-.6) -- (1.5,-0.6);
}
\begin{document}
\begin{tikzpicture}% Example:

    % grille de dessin
    % \draw[color=red] (-10,-10) grid  (10,10);

    % sources
    \begin{scope}[shift={(-.75, 6.5)}]
        \source[Enceinte 2]
    \end{scope}
    \begin{scope}[shift={(-6.5, -.75)}, rotate=90]
        \source[Enceinte 1]
    \end{scope}

    \onde{shift={(0,5.7)}};
    \onde{shift={(-5.7,0)}, rotate=90};

    % pointillés
    \draw[dotted] (0,4.5) -- (0,-2.5);
    \draw[dotted] (-4.5,0) -- (2.5,0);
    \draw[dotted] (-30:-5) -- (-30:2);

    % angle theta
    \draw[thick, ->, >=stealth] (-3,0) arc (0:-30:-3);
    \draw (-15:-3) node[left] {$\theta$};

    % micros
    %% haut
    \draw[fill=gray!40] (-30:-.7) circle (.2);
    \draw (-30:-.9) -- (-30:-1.4);
    \draw (-15:-.95) -- (-45:-.95);
    %% bas
    \draw[fill=gray!40] (-30:.7) circle (.2);
    \draw (-30:.9) -- (-30:1.4);
    \draw (-15:.95) -- (-45:.95);

    % Distances L1 & L2
    \draw[<->,>=stealth,thick] (0,-2) -- (-6.5,-2) node[midway,below] {$L_1$};
    \draw[<->,>=stealth,thick] (2,0) -- (2,6.5) node[midway,right] {$L_2$};

    % textes
    \draw (1,.5) node {Sonde};

    

\end{tikzpicture}
\end{document}
