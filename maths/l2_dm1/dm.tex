\documentclass[a4paper, 11pt]{report} % {{{
	\usepackage[utf8]{inputenc}
	\usepackage[T1]{fontenc}
	\usepackage[french]{babel}
	\usepackage{amsmath}
    \usepackage{geometry}

    \geometry{left=3cm,right=3cm,top=4cm,bottom=3cm}

	% Renuméroter les section /sous sections joliement
    \renewcommand{\thesection}{\Roman{section} - }
    \renewcommand{\thesubsection}{\Alph{subsection}) }
    
    % vecteurs
    \newcommand{\vect}{\overrightarrow}

    \title{L2 SPI --- Maths}
    \author{Mathieu Gaborit}
    \date{Septembre 2012}

% }}}
\begin{document}
    \maketitle

\section{Multiplication de matrices} % {{{

On a les matrices suivantes :

\[
A = \begin{pmatrix}
-1& 2& 5\\
3& 0& 3\\
-5& -2& 0
\end{pmatrix}
~\textrm{et}~
B = \begin{pmatrix}
1& 1\\
2& -1\\
0& 1
\end{pmatrix}
\]

On calcule:

\[
^tA = \begin{pmatrix}
-1& 3& -5\\
2& 0& -2\\
5& 3& 0
\end{pmatrix}
~\textrm{et}~
^tB = \begin{pmatrix}
1& 2& 0\\
1& -1& 1
\end{pmatrix}
\]

Enfin, on calcule $AB$ et $^tB^tA$.

On trouve d'une part :

\[
AB = \begin{pmatrix}
3& 2\\
3& 6\\
-9& -3
\end{pmatrix}
\]

Et d'autre part :

\[
^tB^tA = \begin{pmatrix}
3& 3& -9\\
2& 6& -3
\end{pmatrix}
\]

% }}}
\section{Matrices Inversibles} % {{{

\subsection{}

On a $A = \begin{pmatrix} 1& 0\\ 3& 1 \end{pmatrix}$, on trouve alors $\det{A} = 1$,
ce qui prouve que $A$ est inversible.

On calcule ensuite 
\[
\hat{A} = 
\begin{pmatrix}
1& -3\\
0& 1
\end{pmatrix} \Rightarrow
^t\hat{A} = 
\begin{pmatrix}
1& 0\\
-3& 1
\end{pmatrix}
\]

On trouve enfin :

\begin{eqnarray*}
    A^{-1}  & = & \frac{1}{\det{A}}\cdot^t\hat{A}\\
            & = & \frac{1}{1}\cdot^t\hat{A}\\
            & = & \begin{pmatrix} 1& 0\\ -3& 1 \end{pmatrix}
\end{eqnarray*}

\subsection{}

On a $B = \begin{pmatrix} 1& 0& 1\\ 0& 1& 0\\0&1&1 \end{pmatrix}$, on trouve alors $\det{B} = 1$,
ce qui prouve que $B$ est inversible.

On calcule ensuite 
\[
\hat{B} = 
\begin{pmatrix}
1& 0& 0\\
1& 1& -1\\
-1& 0& 1
\end{pmatrix} \Rightarrow
^t\hat{B} = 
\begin{pmatrix}
1& 1& -1\\
0& 1& 0\\
0& -1& 1
\end{pmatrix}
\]

On remarque de nouveau que $\frac{1}{\det{B}} = 1$, on a :

\[
B^{-1} = ^t\hat{B} = 
\begin{pmatrix}
1& 1& -1\\
0& 1& 0\\
0& -1& 1
\end{pmatrix}
\]

\subsection{}

On a

\[
\begin{vmatrix}
\cos\theta& \sin\theta\\
-\sin\theta& \cos\theta
\end{vmatrix} = \cos^2\theta+\sin^2\theta = 1
\]

On sait donc que $C$ est inversible et on aura $C^{-1} = ^t\widehat{C}$, on calcule :

\begin{eqnarray*}
    C^{-1} = ^t\widehat{C} & = &
    ^t\begin{pmatrix}
        \cos\theta& \sin\theta\\
        -\sin\theta& \cos\theta
    \end{pmatrix}\\
    & = &
    \begin{pmatrix}
        \cos\theta& -\sin\theta\\
        \sin\theta& \cos\theta
    \end{pmatrix}
\end{eqnarray*}

% }}}
\section{Volcan} % {{{

On a : $$p(x,y) = -500 + x^4y^2 + \ln(1+4x^2 + 5y^2)$$

On commence par calculer les dérivées partielles de cette fonction dont nous auront besoin juste après :

\begin{eqnarray*}
    \frac{\partial p}{\partial x} & = & 4x^3y^2 + \frac{8x}{1+4x^2+5y^2}\\
    \frac{\partial p}{\partial y} & = & 2x^4y + \frac{10y}{1+4x^2+5y^2}
\end{eqnarray*}

Les coordonnées du point $Q$ sont $(1,2)$.

\subsection{Direction Nord Ouest en partant de $Q$}

Pour savoir si l'on commence par descendre ou monter, on calcule la différentielle de $p(x,y)$ au point $Q$ :

\begin{eqnarray*}
    dp  & = & \left[4\times2^2 + \frac{8}{1+4+5\times2^2}\right]dx+\left[2\times2 + \frac{10\times2}{1+4+5\times2^2}\right]dy\\
        & = & \left[16+\frac{8}{25}\right]dx+\left[4+\frac{20}{25}\right]dy\\
        & = & \frac{408}{25}dx + \frac{120}{25}dy
\end{eqnarray*}

La "direction nord-ouest" correspond en fait à une infime variation $-\epsilon$ de $dx$ et une variation $+\epsilon$ de $dy$ en
même temps. On injecte ces variations dans l'équation ci-dessus ($\epsilon > 0$):

\begin{eqnarray*}
    dp \pm\epsilon  & = & \frac{408}{25}(dx-\epsilon) + \frac{120}{25}(dy + \epsilon)\\
                    & = & \frac{408}{25}dx + \frac{120}{25}dy + \left(\frac{120}{25} - \frac{408}{25}\right)\epsilon\\
                    & = & dp - \frac{288}{25}\epsilon
\end{eqnarray*}

Ayant pris un $\epsilon$ positif, on a $dp > dp + \epsilon$. Lorsque l'on se déplace dans la direction nord-ouest à
partir du point $Q(1,2)$, on commence par descendre.

\subsection{Direction de plus grande pente}

On a calculé le gradient de $p$ en $Q$ juste au dessus, on a donc : $$\vect{\nabla p} = \frac{1}{25}\cdot\begin{pmatrix} 408\\120 \end{pmatrix}$$

La direction de plus grande pente est la droite dont le vecteur directeur est $\vect{\nabla p(1,2)}$ :

Il s'agit de la droite d'équation :

\[
    \vect{\nabla p(1,2)}_yx - \vect{\nabla p(1,2)}_x y + c = 0
\]

On remplace ensuite les variables par leur valeur :

\[
    \frac{120}{25}x - \frac{408}{25}y + c = 0
\]

et pour trouver le $c$, on regarde ce que donne l'équation au point $Q$ (on est sûrs qu'elle y passe) :

\begin{eqnarray*}
    c & = & \frac{408}{25}\times 2 - \frac{120}{25}\\
      & = & \frac{696}{25}\\
      & = & 27.84
\end{eqnarray*}

La droite de plus grande pente est donc la droite d'équation :

\[
    \frac{1}{25}\left[120x+408y+696\right] = 0
\]

\subsection{Direction pour une pente nulle}

Si on veut une pente nulle, le plus simple est de suivre la ligne de niveau telle que :

\[
    p(x,y) = p(1,2) = -500+4+\ln(25) = \approx -493
\]

On peut aussi décider de considérer comme direction pour une pente nulle la tangente à la ligne de niveau sus-citée en
$Q$. On se retrouve alors à calculer l'équation de la droite ayant pour vecteur normal $\vect{\nabla p(1,2)}$ :

\[
    \vect{\nabla p(1,2)}_xx - \vect{\nabla p(1,2)}_yy + c = 0
\]

Ce qui après substitution nous donne :

\[
    \frac{408}{25}x - \frac{120}{25}y + c = 0
\]

On calcul alors le coefficient $c$ comme précédement :


\begin{eqnarray*}
    c & = & - \frac{408}{25}\times 2 - \frac{120}{25}\\
      & = & -\frac{936}{25}\\
      & = & -37.44
\end{eqnarray*}

On a alors la droite d'équation :

\[
    \frac{1}{25}\left[408x+120y+936\right] = 0
\]

\subsection{Divergence du gradient de $p$}

Ayant calculé les dérivées partielles, on se contente d'écrire :

\[
    \vect{H} = \vect{\nabla p} = \begin{pmatrix}
        4x^3y^2 + \frac{8x}{1+4x^2+5y^2}\\
        2x^4y + \frac{10y}{1+4x^2+5y^2}
    \end{pmatrix}
\]

On calcule ensuite $div\vect{H}$ définie par :

\begin{eqnarray*}
    div\vect{H} & = & \frac{\partial H_x}{\partial x} + \frac{\partial H_y}{\partial y}\\
                & = & 12x^2+y^2 + \frac{8(1+4x^2+5y^2) - 8x\times8x}{\left(1+4x^2+5y^2\right)^2} + 2x^2 + \frac{10(1+4x^2+5y^2) - 10y\times10y}{\left(1+4x^2+5y^2\right)^2}\\
                & = & 2x^2(6y^2+x^2) + \frac{18}{1+4x^2+5y^2} - \frac{64x^2-100y^2}{\left(1+4x^2+5y^2\right)^2}
\end{eqnarray*}
% }}}
=\section{Dérivées partielles d'une fonction composée} % {{{

Soit une fonction $$u(x, y) = x^2 + y^3$$

On a $$x(t, \tau) = e^{t+\tau}$$ et $$y(t, \tau) = \ln(t\tau)$$

On calcule: 

\begin{eqnarray*}
    \frac{\partial u}{\partial t}   & = & \frac{\partial x}{\partial t}\times\frac{\partial f}{\partial x} +
                                                \frac{\partial y}{\partial t}\times\frac{\partial f}{\partial y}\\
                                            & = & e^{t+\tau}\times2e^{t+\tau}+\frac{\tau}{t\tau}3\left[\ln(t\tau)\right]^2\\
                                    & = & 2e^{2t+2\tau} + \frac{3}{t}\left[\ln(t\tau)\right]^2\\
                                    &   &\\
    \frac{\partial u}{\partial\tau}   & = & \frac{\partial x}{\partial\tau}\times\frac{\partial f}{\partial x} +
                                                \frac{\partial y}{\partial\tau}\times\frac{\partial f}{\partial y}\\
                                            & = & e^{t+\tau}\times2e^{t+\tau}+\frac{1}{t\tau}3\left[\ln(t\tau)\right]^2\\
                                    & = & 2e^{2t+2\tau} + \frac{3}{\tau}\left[\ln(t\tau)\right]^2
\end{eqnarray*}

% }}}
\section{Divergence et rotationnel d'un champ de vecteurs} % {{{

On a :

\[
\vect{F} = (z+y)\vect{i} + x\vect{j} + y\vect{k} = \begin{pmatrix}z+y\\x\\y\end{pmatrix}
\]

\subsection{Divergence}

On applique la formule comme suit :

\begin{eqnarray*}
div\vect{F} & = & \frac{\partial F_x}{\partial x} + \frac{\partial F_y}{\partial y} + \frac{\partial F_z}{\partial z}\\
            & = & 0 + 0 + 0 = 0
\end{eqnarray*}

\subsection{Rotationnel}

Calculer $\vect{rot}\vect{F}$ revient à calculer le déterminant suivant en développant par rapport à la première colonne :

\begin{eqnarray*}
\vect{rot}\vect{F}  & = &   \begin{vmatrix}
                                \vect{i} & \frac{\partial}{\partial x} & z + y\\
                                \vect{j} & \frac{\partial}{\partial y} & x\\
                                \vect{k} & \frac{\partial}{\partial z} & y
                            \end{vmatrix}\\
                    & = &   \begin{pmatrix}
                                \frac{\partial y}{\partial y} - \frac{\partial x}{\partial z}\\
                                -\frac{\partial y}{\partial x} + \frac{\partial (z+y)}{\partial z}\\
                                \frac{\partial x}{\partial x} - \frac{\partial (z+y)}{\partial y}
                            \end{pmatrix}\\
                    & = &   \begin{pmatrix}
                        1 - 0\\
                        0 + 1\\
                        1 - 1
                    \end{pmatrix}\\
                    & = &   \begin{pmatrix}
                        1\\
                        1\\
                        0
                    \end{pmatrix}
\end{eqnarray*}
 

% }}}
\section{Champ, potentiel, divergence et rotationnel, nos amis de toujours\ldots} % {{{

On a en coordonnées cylindriques :

$$V(r, \theta, z) = \frac{1}{1+(z+r\cos\theta)^2} + \frac{1}{1+(z+r\sin\theta)^2}$$

En coordonnées cartésiennes, on arrive à :

$$V(x,yz) = \frac{1}{1+(z+x)^2} + \frac{1}{1+(z+y)^2}$$

\subsection{Calcul du gradient en coordonnées cylindriques}

On commence par le calcul des dérivées partielles :

\begin{eqnarray*}
    \frac{\partial V}{\partial r}   & = & -\frac{2(z+r\cos\theta)\cos\theta}{\left[1+(z+r\cos\theta)^2\right]^2} - \frac{2(z+r\sin\theta)\sin\theta}{\left[1+(z+r\sin\theta)^2\right]^2}\\
    \frac{\partial V}{\partial \theta}   & = & \frac{2(z+r\cos\theta)r\sin\theta}{\left[1+(z+r\cos\theta)^2\right]^2} - \frac{2(z+r\sin\theta)r\cos\theta}{\left[1+(z+r\sin\theta)^2\right]^2}\\
    \frac{\partial V}{\partial z}   & = & -\frac{2(z+r\cos\theta)}{\left[1+(z+r\cos\theta)^2\right]^2} - \frac{2(z+r\sin\theta)}{\left[1+(z+r\sin\theta)^2\right]^2}\\
\end{eqnarray*}

On peut ensuite écrire le gradient de $V$ en coordonnées cylindriques :

$$\vect{\nabla V} = \vect{E} = \frac{2(z+r\cos\theta)}{\left[1+(z+r\cos\theta)^2\right]^2}
\begin{pmatrix}
    -\cos\theta\\
    r\sin\theta\\
    -1
\end{pmatrix}
- \frac{2(z+r\sin\theta)}{\left[1+(z+r\sin\theta)^2\right]^2}
\begin{pmatrix}
    \sin\theta\\
    r\cos\theta\\
    1
\end{pmatrix}
$$

\subsection{Calcul du gradient en coordonnées cartésiennes}

De nouveau, on commence par calculer les dérivées partielles de $V(x,y,z)$ :

\begin{eqnarray*}
    \frac{\partial V}{\partial x} & = & -\frac{2(z+x)}{\left[1+(z+x)^2\right]^2}\\
    \frac{\partial V}{\partial y} & = & -\frac{2(z+y)}{\left[1+(z+y)^2\right]^2}\\
    \frac{\partial V}{\partial z} & = & -\frac{2(z+x)}{\left[1+(z+x)^2\right]^2} - \frac{2(z+y)}{\left[1+(z+y)^2\right]^2}
\end{eqnarray*}


On peut alors écrire le gradient de $V(x,y,z)$ :

$$\vect{\nabla V} = \vect{E} = -\frac{2(z+x)}{\left[1+(z+x)^2\right]^2}
\begin{pmatrix}
    1\\
    0\\
    1
\end{pmatrix}
- \frac{2(z+y)}{\left[1+(z+y)^2\right]^2}
\begin{pmatrix}
    0\\
    1\\
    1
\end{pmatrix}
$$

\subsection{Divergence}

On calcule $\frac{\partial E_x}{\partial x}$ et $\frac{\partial E_y}{\partial y}$ :

\begin{eqnarray*}
    \frac{\partial E_x}{\partial x} & = & \frac{-2\left[1+(z+x)^2\right]^2 - 4\left[1+(z+x)^2\right](z+x)}{\left[1+(z+x)^2\right]^4}\\
                                    & = & \frac{-2\left[\left[1+(z+x)^2\right]-2(x+z)\right]}{\left[1+(z+x)^2\right]^3}\\
    \frac{\partial E_y}{\partial y} & = & \frac{-2\left[1+(z+y)^2\right]^2 - 4\left[1+(z+y)^2\right](z+y)}{\left[1+(z+y)^2\right]^4}\\
                                    & = & \frac{-2\left[\left[1+(z+y)^2\right]-2(y+z)\right]}{\left[1+(z+y)^2\right]^3}
\end{eqnarray*}

On remarque que $\frac{\partial E_z}{\partial z} = \frac{\partial E_x}{\partial x} +\frac{\partial E_y}{\partial y}$ :

$$
    \frac{\partial E_z}{\partial z} = \frac{-2\left[\left[1+(z+x)^2\right]-2(x+z)\right]}{\left[1+(z+x)^2\right]^3} + \frac{-2\left[\left[1+(z+y)^2\right]-2(y+z)\right]}{\left[1+(z+y)^2\right]^3}
$$

La divergence de $\vect{E}$, c'est la somme de ces trois termes, on a donc :

$$div\vect{E} = 2\frac{\partial E_x}{\partial x} + 2\frac{\partial E_y}{\partial y}$$

Ce qui nous donne

$$
div\vect{E} = \frac{-4\left[\left[1+(z+x)^2\right]-2(x+z)\right]}{\left[1+(z+x)^2\right]^3} + \frac{-4\left[\left[1+(z+y)^2\right]-2(y+z)\right]}{\left[1+(z+y)^2\right]^3}
$$

\subsection{Rotationnel}

Calculer le rotationnel de $\vect{E}$, revient à calculer le déterminant de la matrice suivant en développant par
rapport à la première colonne :

$$
\vect{rot}\vect{E} =\begin{vmatrix}
                        \vect{i} & \frac{\partial}{\partial x} & E_x\\
                        \vect{j} & \frac{\partial}{\partial y} & E_y\\
                        \vect{k} & \frac{\partial}{\partial z} & E_z
                    \end{vmatrix}\\
$$

On remarque que $E_y$ ne dépendant pas de $x$, $\frac{\partial E_y}{\partial x} = 0$, il en va de même pour $E_x$ et
$y$.
On a alors : $\frac{\partial E_y}{\partial x} - \frac{\partial E_x}{\partial y} = 0$.


Pour le calcul de $\frac{\partial E_z}{\partial x} - \frac{\partial E_x}{\partial z}$, on remarque seule la première
partie de $E_z$ dépend de $x$ et que celle ci est égale à $E_x$. Le calcul à effectuer de vient donc $\frac{\partial
E_x}{\partial x} - \frac{\partial E_x}{\partial z}$ Et dans $E_x$, $x$ et $z$ ont des rôles symétriques, on peut donc
écrire :

$$
\frac{\partial E_x}{\partial x} = \frac{\partial E_x}{\partial z}
$$
et
$$
\frac{\partial E_x}{\partial x} - \frac{\partial E_x}{\partial z} = 0
$$

Enfin, on remarque que l'on peut appliquer le même raisonnement pour $\frac{\partial E_y}{\partial z}$ et
$\frac{\partial E_z}{\partial y}$

On écrit alors : 
$$
\frac{\partial E_y}{\partial y} - \frac{\partial E_y}{\partial z} = 0
$$

Et on déduit de cela le rotationnel de $\vect{E}$ :

$$\vect{rot}\vect{E} = \begin{pmatrix}0\\0\\0\end{pmatrix}$$

% }}}


\end{document}
