\documentclass[a4paper, 11pt]{article}
	
	\usepackage[utf8]{inputenc}
	\usepackage[T1]{fontenc}
	\usepackage[french]{babel}
	\usepackage{amsmath}

	\title{Mémo Acoustique}
	\author{L1SPI\\
        Largement inspiré de l'initiation à l'acoustique}
	\date{Mai 2012}

\begin{document}
	\maketitle

    \section{Chapitre 1}

    Ondes :
    \begin{description}
        \item[longitudinale] Direction de la perturbation est parallèle à celle de la propagation
        \item[transversale] Direction de la perturbation est perpendiculaire à celle de la propagation
    \end{description}

    Célérité d'une manière générale :
    $$c = \sqrt{\frac{param. de raideur}{param. d'inertie}}$$

    Définitions autour de la célérité et de la vitesse particulaire

    \begin{description}
        \item[Particule] Volume de matière assez grand pour négliger l'effet moléculaire mais suffisament petit par rapport à la longueur d'onde pour que les grandeurs puissent être considérées uniformes dans la particule.
        \item[vitesse de propagation] Vitesse de transmission du phénomène ondulatoire
        \item[vitesse particulaire] vitesse de vibration d'une particule au passage de la perturbation
    \end{description}

    Fréquence et longueur d'onde :
    $$\lambda = \frac{c}{f}$$

    Plage audible dans l'air :
    $$20Hz \leq f \leq 20kHz$$


    \section{Chapitre 2}

    \subsection{Valeur Moyenne DC}

    Série de $N$ valeur $S_i$ :
    $$\bar{S} = \frac{1}{N}\sum_{i=1}^NS_i$$

    Signal discret $S(t)$ :
    $$\bar{S} = \frac{1}{N\Delta t}\sum_{i=1}^NS(i)\Delta t$$

    Signal continu $s(t)$ :
    $$\bar{S} = \frac{1}{T}\int_Ts(t)dt$$

    Energie :
    $$E = \int_T [s(t)]^2dt$$

    Puissance :
    $$P = \frac{1}{T}\int_T[s(t)]^2dt$$

    Valeur RMS :
    $$s_{RMS} = \sqrt{\frac{1}{T}\int_T[s(t)]^2dt}$$
    
    Valeur ACRMS :
    $$s_{ACRMS} = \sqrt{\frac{1}{T}\int_T[s(t)-\bar{s}]^2dt}$$

    Périodicité et fréquence :
    \begin{itemize}
        \item signal est périodique si $$s(t+T) = s(t)$$
        \item La plus petite valeur possible pour $T$ définit la période
        \item Lien entre période et fréquence : $$f=\frac{1}{T}$$
    \end{itemize}

    Expression d'un signal monochromatique :
    $$s(t) = A\cos(\omega t+\phi)$$
    \begin{itemize}
        \item $A$ l'amplitude
        \item $\omega=2\pi f$ : pulsation
        \item $\phi$ la phase
    \end{itemize}

    \section{Chapitre 3}

    Intervalle en les notes :
    \begin{description}
        \item[Octave] facteur 2
        \item[Tiers d'octave] facteur $\sqrt[3]{2}$
        \item[Demi-ton] facteur $\sqrt[12]{2}$
    \end{description}

    Niveau en dB :
    $$L_U = 20\log_{10}\left(\frac{U_{RMS}}{U_0}\right)$$

    En dBSPL (avec $I_0=1\cdot10^{-12}W/m^2$ et $p_0=2\cdot10^{-5}Pa$) :
    $$L_I = 10\log_{10}\left(\frac{I_{RMS}}{I_0}\right)$$
    $$L_p = 20\log_{10}\left(\frac{p_{RMS}}{p_0}\right)$$


\end{document}
