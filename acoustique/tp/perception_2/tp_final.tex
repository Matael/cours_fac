\documentclass[a4paper]{report}
% generated by Docutils <http://docutils.sourceforge.net/>
\usepackage{fixltx2e} % LaTeX patches, \textsubscript
\usepackage{cmap} % fix search and cut-and-paste in Acrobat
\usepackage{ifthen}
\usepackage[T1]{fontenc}
\usepackage[utf8]{inputenc}
\usepackage{color}
\usepackage{longtable,ltcaption,array}
\setlength{\extrarowheight}{2pt}
\newlength{\DUtablewidth} % internal use in tables

%%% Custom LaTeX preamble
% PDF Standard Fonts
\usepackage{mathptmx} % Times
\usepackage[scaled=.90]{helvet}
\usepackage{courier}
\usepackage{geometry}

\geometry{top=2cm,right=1.5cm,bottom=3cm,left=1.5cm}
\usepackage{amsmath}
\usepackage{amsfonts}

%%% User specified packages and stylesheets

%%% Fallback definitions for Docutils-specific commands

% admonition (specially marked topic)
\providecommand{\DUadmonition}[2][class-arg]{%
  % try \DUadmonition#1{#2}:
  \ifcsname DUadmonition#1\endcsname%
    \csname DUadmonition#1\endcsname{#2}%
  \else
    \begin{center}
      \fbox{\parbox{0.9\textwidth}{#2}}
    \end{center}
  \fi
}

% title for topics, admonitions and sidebar
\providecommand*{\DUtitle}[2][class-arg]{%
  % call \DUtitle#1{#2} if it exists:
  \ifcsname DUtitle#1\endcsname%
    \csname DUtitle#1\endcsname{#2}%
  \else
    \smallskip\noindent\textbf{#2}\smallskip%
  \fi
}

% hyperlinks:
\ifthenelse{\isundefined{\hypersetup}}{
  \usepackage[colorlinks=true,linkcolor=blue,urlcolor=blue]{hyperref}
  \urlstyle{same} % normal text font (alternatives: tt, rm, sf)
}{}
\hypersetup{
  pdftitle={Sources Cohérentes, Battements, Sources Incohérentes},
}

%%% Title Data
\title{\phantomsection%
  Sources Cohérentes, Battements, Sources Incohérentes%
  \label{sources-coherentes-battements-sources-incoherentes}}
\author{Erwan Marchand \and Mathieu Gaborit}
\date{Mai 2012}
    
%%% Table of contents

\setcounter{tocdepth}{4}
\renewcommand{\contentsname}{Au menu}

%%% Body
\begin{document}
\maketitle

\tableofcontents
\newpage

Cette séance de travaux pratiques avait pour objectif l'étude de l'addition de sources sonores (en fonction de leurs fréquences et phases respectives) ainsi que celle des effets perceptifs qui y sont liés.

Pour ce faire, nous procéderons en deux parties.

Nous nous intéresseront premièrement aux sources cohérentes, avec quelques démonstrations théoriques ainsi que deux manipulations.

La seconde partie portera sur les phénomènes de battements et les sources incohérentes.
Nous traiterons d'abord de théorie, puis passerons à l'expérimentation.

Ce TP sera aussi marqué par quelques références à des applications et des tracés via Octave.


%___________________________________________________________________________

\section*{\phantomsection%
  Sources Cohérentes%
  \addcontentsline{toc}{section}{Sources Cohérentes}%
  \label{sources-coherentes}%
}

L'objectif est d'étudier l'influence du déphasage sur le signal résultant de l'addition de deux sources.


%___________________________________________________________________________

\subsection*{\phantomsection%
  Théorie%
  \addcontentsline{toc}{subsection}{Théorie}%
  \label{theorie}%
}

On a
\begin{eqnarray*}
s_1 & = & A\sin(2\pi ft)\\
s_1 & = & A\sin(2\pi ft + \phi)
\end{eqnarray*}

Ce qui nous donne :

\begin{eqnarray*}
V_{s_1eff} & = & \frac{A}{\sqrt{2}}\\
V_{s_2eff} & = & \frac{A}{\sqrt{2}}
\end{eqnarray*}

Pour la somme, on a:

\begin{eqnarray*}
s_1+s_2 & = & A\left[ \sin(2\pi ft) + \sin(2\pi ft+\phi)\right]
\end{eqnarray*}

On peut alors calculer $V_{eff} = V_{(s_1+s_2)eff}$ :

\begin{eqnarray*}
V_{eff} & = & \sqrt{\frac{1}{T}\int_TA^2\left[\sin(2\pi ft) + \sin(2\pi ft + \phi)\right]^2dt}\\
V_{eff}^2 & = & \frac{A^2}{T}\int_T\left[\sin(2\pi ft) + \sin(2\pi ft + \phi)\right]^2dt\\
\frac{T}{A^2}\cdot V_{eff}^2 & = & \int_T\left[\sin(2\pi ft) + \sin(2\pi ft + \phi)\right]^2dt\\
& = & \frac{1}{2}\int_T\cos(0)dt - \frac{1}{2}\int_T\cos(4\pi ft)dt+\frac{1}{2}\int_T\cos(0)dt-\frac{1}{2}\int_T\cos(4\pi ft+\phi)dt + \int_T\cos(-\phi)dt -\int_T(4\pi ft+\phi)dt\\
\frac{T}{A^2}V_{eff}^2 & = & \int_T\cos(0)dt + \int_T\cos(\phi)dt\\
V_{eff}^2 & = & \frac{A^2}{T}\cdot T + \frac{A^2}{T} \cdot T\cos(\phi)\\
V_{eff}^2 & = & A^2 + A^2\cdot\cos(\phi)
\end{eqnarray*}


Il est alors possible de calculer le gain en dB. Pour cela, on utilisera une formule plus générale :

\begin{eqnarray*}
G & = & 10\log_{10}\left(\frac{A^2+A^2\cos(\phi)}{\frac{A^2}{2}}\right)\\
& = & 10\log_{10}\left(\frac{A^2+A^2\cos(\phi)}{A^2}\right) + 10\log_{10}(2)\\
& = & 10\log_{10}(\cos(\phi)+1) + 3
\end{eqnarray*}


On a alors :

\begin{center}
\begin{tabular}{|c|c|}\hline
$G(0)$ & $+6dB$\\\hline
$G(\frac{\pi}{2})$ & $+3dB$\\\hline
$G(\pi)$ & $-\infty dB$\\\hline
\end{tabular}
\end{center}



%___________________________________________________________________________

\subsection*{\phantomsection%
  Expérience%
  \addcontentsline{toc}{subsection}{Expérience}%
  \label{experience}%
}


%___________________________________________________________________________

\subsubsection*{\phantomsection%
  Analyse subjective%
  \addcontentsline{toc}{subsubsection}{Analyse subjective}%
  \label{analyse-subjective}%
}

Après avoir relié les appareils comme demandé, nous réalisons la série suivante :

\begin{center}
\begin{tabular}{|c|c|}\hline
Fréquence  & Déphasage pour un min. ($k\in\mathbb{Z}=$)\\\hline
 100 Hz    &        $\pi +2k\pi$        \\
 200 Hz    &        $\pi +2k\pi$        \\
 800 Hz    &        $\pi +2k\pi$        \\
1600 Hz    &        $\pi +2k\pi$        \\\hline
\end{tabular}
\end{center}

On note que c'est en adéquation avec les calculs.

Nous observons aussi des maxima pour 0.
Là encore, c'est en adéquation avec la théorie.

Les maxima et minima se ``sentent'' bien à l'oreille.


%___________________________________________________________________________

\subsubsection*{\phantomsection%
  Utilisation d'un microphone%
  \addcontentsline{toc}{subsubsection}{Utilisation d'un microphone}%
  \label{utilisation-d-un-microphone}%
}

Après avoir fixé les amplitudes des deux signaux à l'oscilloscope, nous mesurons.

Pour une source seule, nous avons :

$$Vrms = 35 mV$$

Cette valeur nous servira de référence dans le calcul des gains.

Pour deux sources en phase :

$$Vrms = 45mV$$

Ce qui nous donne un gain de 

$$G = 20log(45/35) = 2.18dB$$

Pour deux sources en quadrature de phase :

$$Vrms = 40 mV$$

Ce qui nous donne un gain de 

$$G = 20log(40/35) = 1.15dB$$

Pour deux sources en opposition de phase :

$$Vrms = 14mV$$

Ce qui nous donne un gain de

$$G = 20log(14/35) = -7.95dB$$

On peut résumer le tout ainsi :

\begin{center}
\begin{tabular}{|c|c|c|}\hline
Déphasage     &     Gain Théorique    &  Gain expérimental   \\\hline
  $0$          &         $+6dB$         &       $+2.18dB$        \\
  $\frac{\pi}{2}$       &         $+3dB$         &       $+1.15dB$        \\
  $pi$         &         $-infty$       &       $-7.95dB$        \\\hline
\end{tabular}
\end{center}

Même si l'adéquation n'est pas parfaite, on remarque que le rapport entre les gains pour $\phi=0$ et $\phi=\frac{\pi}{2}$ est quasiment conservé et que celui pour $\phi=\pi$ est largement en deçà des deux autres.

Les causes possibles d'inadéquation entre pratique et théorie peuvent être :
%
\begin{itemize}

\item mauvaise qualité de l'ampli, des haut-parleurs

\item mauvaise précision dans le règlage

\item caractère non-idéal de la chaine d'excitation

\item mauvais positionnement des sources, de la prise de mesure

\item phénomènes liés à l'environnement (réverbération, etc...)

\end{itemize}


%___________________________________________________________________________

\subsubsection*{\phantomsection%
  Applications%
  \addcontentsline{toc}{subsubsection}{Applications}%
  \label{applications}%
}

Pour ce qui est de la conduite, on peut envisager de place un micro dans la conduite.
Le signal capté est ensuite déphasé de pi et renvoyé vers un haut parleur situé dans cette même conduite, après le micro.

L'amélioration suivante pourrait être d'utiliser un système d'asservissement et un second micro en sortie de conduite afin de mieux contrôler les effets.

Le bouton PHASE applique un déphasage de $\pi$ au signal pour que celui ci revienne correctement.
Si on applique ce même déphasage deux fois, on obtiendra un déphasage résultant de 0 ce qui produira l'effet inverse : un amplification du signal.


%___________________________________________________________________________

\section*{\phantomsection%
  Battements - Sources Incohérentes%
  \addcontentsline{toc}{section}{Battements - Sources Incohérentes}%
  \label{battements-sources-incoherentes}%
}


%___________________________________________________________________________

\subsection*{\phantomsection%
  Avant%
  \addcontentsline{toc}{subsection}{Avant}%
  \label{avant}%
}

La somme est un sinus dont l'amplitude varie de $-2A$ à $2A$ (avec A l'amplitude de l'excitation initiale).

D'après l'affichage sur l'oscilloscope, on peut prévoir d'entendre une sorte ``d'ondulation en amplitude''.

Au cours de la rédaction du compte rendu, nous avons tracé le phénomène avec Octave, ce tracé est disponible en annexe 1.

La visualisation du profil de pression à l'oscilloscope (via un microphone) est en adéquation avec ce que nous avions modélisé précédement.
Les légères différences perceptibles sont probablement dûes au caractère non idéal (et loin de là) de l'environnement (réverbération, bruit de fond non négligeable, etc...).


%___________________________________________________________________________

\subsection*{\phantomsection%
  Impressions%
  \addcontentsline{toc}{subsection}{Impressions}%
  \label{impressions}%
}

On ressent une ``ondulation'' toutes les secondes environ.

Nos deux fréquences utilisées sont 100Hz et 101Hz (cf. modélisation en annexe 1).

L'oscilloscope indique sensiblement la même chose : une fréquence de battement de 1Hz.

Afin de comprendre (ou du moins d'essayer) l'influence du $\Delta f$ sur le phénomène de battement, nous avons aussi essayé de tracer les courbes résultantes pour plusieurs $\Delta f$. Ces courbes sont disponibles en annexe 3.

On remarque que lorsque $\Delta f$ devient grand ($\approx$ 50Hz) le signal résultant se comporte comme l'addition de 2 sources incohérentes.
Pour des $\Delta f$ moyens (entre 5 et 40Hz environ), l'interférence entre les sources est importante et à l'oreille, le battement est très présent.
Pour des $\Delta f$ petits (moins de 5Hz), le signal résultant est affecté, mais cela n'est pas particulièrement dérangeant à l'oreille.

La fréquence des battements augmente progressivement puis ceux ci disparaissent.


%___________________________________________________________________________

\subsection*{\phantomsection%
  Mesures Quantitatives%
  \addcontentsline{toc}{subsection}{Mesures Quantitatives}%
  \label{mesures-quantitatives}%
}

A l'oreille, on compte environ 1 battement par seconde, or,

$$\Delta f = f_1 - f_2 = 1 Hz$$

En augmentant progressivement $f_2$, on finit par ne plus entendre le phénomène. La fréquence limite est de 150Hz (ce que nous avions remarqué en annexe 3).

Pour différents $f_1$ (200Hz, 400Hz, 1kHz et 2kHz), on remarque que la fréquence limite est souvent telle que $f_lim = f_1 + 50Hz$ .

Si les fréquences sont multiples, alors on observe un signal de la plus grande des deux fréquence modulé en offset par le signal de la plus faible fréquence.


%___________________________________________________________________________

\subsection*{\phantomsection%
  Application%
  \addcontentsline{toc}{subsection}{Application}%
  \label{application}%
}

L'accord d'un instrument peut se faire en tirant parti de ce phénomène.

On joue par exemple une note sur la corde à accorder en même temps qu'une note ``étalon'' dont on sait qu'elle est juste.
On accorde alors en cherchant à diminuer la fréquence des battements. Lorsque l'on parvient à les annuler, on a atteint un multiple de la fréquence étalon, ensuite, il faut faire confiance à son oreille pour être sûr de ne pas s'être trompé d'octave...

\bigskip

On observe deux type d'interférence dûes au déphasage, les interférences contructives lorsque les haut-parleurs sont déphasés de $2k\pi$ et les interférences destructives lorsqu'ils sont déphasés de $(2k-1)\pi$ ($k\in\mathbb{Z}$).

Entre les deux on observe un phase de transition.

On observe également un phénomène de battement lorsque deux sources émettent avec des fréquence proches.
De plus, les phénomènes étudiés ici ont été mis en relation avec des cas concrets, ce qui nous permit de mieux en saisir l'utilité.

\end{document}
