\documentclass[a4paper, 11pt]{article}
	
	\usepackage[utf8]{inputenc}
	\usepackage[T1]{fontenc}
	\usepackage[french]{babel}
	\usepackage{amsmath}
    \usepackage[top=2cm,bottom=2cm,right=2cm,left=2cm]{geometry}

	\title{Mémo Electro-magnétisme}
	\author{L2 SPI}
	\date{Janvier 2013}

    \newcommand{\ve}{\overrightarrow}

    % intégrale de flux :
    \newcommand{\fint}{%
        \setbox0=\hbox{$\displaystyle \int\!\!\!\!\!\int$}
        \setbox1=\hbox to \wd0{\hfill$\bigcirc$\hfill}
        \setbox0=\hbox to \wd0{\copy0\hss\copy1}
        \mathop{\copy0}
    }
    \newcommand{\tint}{\int\!\!\!\!\!\int\!\!\!\!\!\int}
    \newcommand{\dint}{\int\!\!\!\!\!\int}
    \newcommand{\dv}{\mathrm{div}}


\begin{document}
	\maketitle

    \section{Rappels d'électro-statique}

    \subsection{Loi de Coulomb}

    $$\ve{E} = \frac{1}{4\pi\epsilon_0}\cdot\frac{q}{r^2}\cdot\ve{u_R}$$

    Avec :
    \begin{itemize}
        \item $r$ la distance à la particule chargée
        \item $\ve{u_R}$ le vecteur unitaire sur l'axe entre la charge et le point où l'on calcule
        \item $\epsilon_0$ la permitivité diélectrique du vide
    \end{itemize}

    \subsection{Théorème de Gauss}

    $$\Phi = \fint_{S}\ve{E}\cdot\ve{\mathrm{d}S} = \frac{\sum_{q_{int}}}{\epsilon_0}$$

    \section{Magnétostatique}

    \subsection{Loi de Biot et Savart}

    $$\ve{B} = \frac{\mu_0}{4\pi}\oint_{(C)}\frac{I\ve{\mathrm{d}l}\wedge\ve{PM}}{PM^3}$$

    Avec :
    \begin{itemize}
    \item $\mu_0$ la perméabilité magnétique du vide
    \end{itemize}

    \section{Electro-magnétisme}

    \subsection{Théorème d'Ampère}

    $$\Phi = \oint\ve{B}\cdot\ve{\mathrm{d}l} = \mu_0\sum_iI_i$$

    Avec $\ve{\mathrm{d}l}$ suivant le parcours de $I$.

    \subsection{Force de Lorentz}

    $$\ve{F} = q\left(\ve{E}+\ve{v}\wedge\ve{B}\right)$$

    Avec $\ve{v}$ la vitesse de la particule $q$.

    \subsection{Force de Laplace}

    $$\ve{F} = I\oint_{(C)}\ve{dr}\wedge\ve{B}$$

    Avec $(C)$ le contour du fil baignant dans $\ve{B}$.

    \subsection{Théorème de la divergence}

    $$\fint_\Sigma\ve{A}\cdot\ve{\mathrm{d}S} = \tint_V\dv\ve{A}\mathrm{d}V$$

    Avec :
    \begin{itemize}
        \item $V$ un volume
        \item $\Sigma$ la surface entourant ce volume
    \end{itemize}

    \subsection{Théorème de Stokes}

    $$\oint_C\ve{A}\cdot\ve{\mathrm{d}l} = \dint_{\Sigma}\ve{\mathrm{rot}}\ve{A}\cdot\ve{\mathrm{d}S}$$
    Avec :
    \begin{itemize}
        \item $\Sigma$ une surface
        \item $C$ le contour de cette surface
    \end{itemize}

    \subsection{Loi de Faraday}

    $$ e = - \frac{\mathrm{d}\Phi}{\mathrm{d}t}$$

    Avec :
    \begin{itemize}
        \item $e$ la force électromotrice
    \end{itemize}

    \subsection{Lien champ$\leftrightarrow$ potentiel}

    $$\ve{E} = -\ve{\mathrm{grad}}{e}$$

    Avec :
    \begin{itemize}
        \item $\ve{E}$ le champ électrique
        \item $e$ le potentiel
    \end{itemize}

    En découle par exemple :

    $$e = \int \ve{E}\cdot\ve{\mathrm{d}l}$$

    \section{Equation de Maxwell}

    \subsection{Equation de Maxwell-Gauss}

    $$\dv\ve{D} = \rho$$

    Avec :
    \begin{itemize}
        \item $\ve{D}$ le vecteur déplacement électrique
        \item $\rho$ la charge volumique
    \end{itemize}

    \subsection{Equation de Maxwell-Ampère}

    $$\ve{\mathrm{rot}}\ve{H} = \ve{J_c} + \frac{\partial\ve{D}}{\partial t}$$

    Avec :
    \begin{itemize}
        \item $\ve{H}$ vecteur excitation magnétique
        \item $\ve{J_c}$ vecteur densité de courant
    \end{itemize}

    \subsection{Equation de Maxwell-Thomson}

    $$\dv\ve{B} = 0$$

    Avec :
    \begin{itemize}
        \item $\ve{B}$ le champ magnétique
    \end{itemize}

    \subsection{Equation de Maxwell-Faraday}

    $$\ve{\mathrm{rot}}\ve{E} = -\frac{\mathrm{d}\ve{B}}{\mathrm{d}t}$$

    \subsection{Equations générales -- Equations dans le vide}

    \begin{tabular}{|c|c|c|}
    \hline
    & Générales & Dans le vide\\\hline
    Maxwell-Gauss & $\dv(\epsilon_0\ve{E}) = \rho$ & $\dv(\epsilon_0\ve{E}) = 0$\\\hline
    Maxwell-Ampère & $\ve{\mathrm{rot}}\left(\frac{\ve{B}}{\mu_0}\right)-\frac{\partial(\epsilon_0\ve{E})}{\partial t} = \ve{j}$ & $\ve{\mathrm{rot}}\left(\frac{\ve{B}}{\mu_0}\right)-\frac{\partial(\epsilon_0\ve{E})}{\partial t} = \ve{0}$ \\\hline
    Maxwell-Faraday & \multicolumn{2}{c|}{$\ve{\mathrm{rot}}\ve{E} = -\frac{\partial\ve{B}}{\partial t}$}\\\hline
    Maxwell-Thomson & \multicolumn{2}{c|}{$\dv\ve{B} = 0$}\\\hline
    \end{tabular}
\end{document}
