\section{Impression personnelle}

\subsection{Sur l'organisation}

Du côté de l'organisation, je dois admettre que je suis un peu déçu... et en même temps, pas autant que je pourrais
l'être.

Tout d'abord, nous avons eu l'impression (au moins Baptiste et moi-même) d'être un peu livrés à nous-même. La séance se
passait souvent ainsi : nous arrivions, attendions que les professeurs arrivent (la ponctualité commence cinq minutes avant
l'heure paraît-il...), prenions place dans une salle, recevions un groupe d'élèves en même temps que le sujet du jour et
nous débrouillions pour faire un cours sur le sujet (une feuille d'exercices nous étant parfois fournie).

Il y a eu des semaines et des notions pour lesquelles il n'y avait aucun problème, d'autres où il fallait faire preuve
de \textit{beaucoup} d'inventivité et d'adaptabilité...

Je suis déçu pour cette raison particulièrement et en même temps, cela m'a permit d'apprendre la transmission de
connaissances "sans filet". Pas de préparation de cours possible, pas de \textit{slides} ou de fiche préparée, juste un tableau
et les connaissances apprises des années auparavant. J'avoue avoir beaucoup apprécié cet aspect de l'UEL.

% jurys
J'ai aussi trouvé très intéressants les jurys de soutenance. Je suis convaincu que passer de "l'autre côté du bureau" me
permettra d'être moi même plus efficace lors de mes futures présentations.

Cette expérience était importante aussi en ce qu'elle nous a permis d'aborder à la fois la partie "soutenance" et la
partie "notation". Il ne s'agissait pas de sanctionner les élèves mais de mettre une note seulement si celle-ci était
supérieure à la moyenne. Je ne cautionnais (ni ne cautionne aujourd'hui) cette pratique, mais je comprends désormais le
point de vue des enseignants.

Finalement, ces mini-TPE m'ont fait relativiser ma vision de la notation, le niveau de rendu que l'on peut attendre d'un
élève, mais aussi prendre conscience de combien les attentes avaient changées depuis la fin de mon lycée.

\subsection{Sur l'UEL}

Cette UEL m'a permis de reprendre contact avec le monde de l'enseignement. Dans ma vie associative, j'ai plusieurs fois
dispensé des formations (sur des aspects plus techniques, pour un public plus âgé) ou des présentations. Ici, j'ai pu
retrouver l'expérience d'une classe, avec un programme que je ne définissais pas et pour expliquer des notions qui dataient
parfois d'assez loin.

Globalement, j'ai dû me souvenir des "astuces" pour faire passer certaines notions, inventer des exemples à la volée,
apprendre à me mettre au niveau des apprenants (chose qui n'est pas forcément importante lorsque votre public est plus
âgé et n'hésite pas à poser des questions).

J'ai en partie appris à faire apprendre (je ne prétends pas enseigner).

Si je devais résumer rapidement, j'écrirais que cette UEL m'a montré toute la difficulté de gérer un groupe et de faire
avancer celui-ci sans perdre personne. Je comprends mieux la problématique d'enseignement et la tâche qu'un professeur
doit accomplir.

Par ailleurs, les discussions avec les différents professeurs m'ont aussi fait remarquer qu'il s'agit d'un milieu où la
prise de recul semble être compliquée. En effet, ceux ci voient leurs élèves tous les jours, et peuvent difficilement
prendre du temps pour réfléchir à la meilleure approche à adopter pour un concept donné avec une classe donnée.

Enfin, il faut savoir que les élèves qui venaient nous voir étaient pour une majorité contraints. Certains ne
présentaient pas de réelles difficultés mais plus un manque de confiance en eux que seul un travail en petit groupe permet de
pallier. D'autres enfin ne cherchaient pas à comprendre presque par défi, comme pour prouver à l'enseignant qu'il ne les
changerait pas... ce pan de la psychologie de l'élève (que j'avais déjà expérimenté en cours particulier) explique
probablement la démoralisation d'une part du corps enseignant...
