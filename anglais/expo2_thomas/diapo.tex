\documentclass{beamer}

    \usepackage[utf8]{inputenc}
    \usepackage[T1]{fontenc}
    \usepackage[french]{babel}
    \usepackage{url}

    \usetheme{Warsaw}

    \setbeamertemplate{navigation symbols}{}%remove navigation symbols

    % Faire apparaître un sommaire avant chaque section
    \AtBeginSection[]{
        \begin{frame}
            %%% affiche en début de chaque section, les noms de sections et
            %%% noms de sous-sections de la section en cours.
            \tableofcontents[currentsection]
        \end{frame} 
    }

    % Faire apparaître un sommaire avant chaque section
    \AtBeginSubsection[]{
        \begin{frame}
            %%% affiche en début de chaque section, les noms de sections et
            %%% noms de sous-sections de la section en cours.
            \tableofcontents[currentsection,currentsubsection]
        \end{frame} 
    }

    \title[How to get a water drop levitating]{How to get a water drop levitating}
    \institute{Université du Maine}
    \author{Thomas Lechat \and Mathieu Gaborit}
    \date{Novembre 2012}

\begin{document}

\begin{frame}
\titlepage
\end{frame}

\section{2 Ways}
\begin{frame}
\frametitle{2 ways}

\begin{itemize}
    \item optical illusion
    \item real levitation (gravity compensation)
\end{itemize}

\end{frame}

\section{Optical Illusions}
\subsection{Optical Illusion 1 : Change frame reference}
\begin{frame}
\frametitle{Optical Illusion 1 : Change frame reference}

In mechanics, all is about frame reference.

The water drop has to levitate from the observer's point of view

\pause
\begin{center}
sooo....
\end{center}

\pause

\begin{center}
Drop the observer and the water drop at once !
\end{center}

\pause

If he falls as rapidly as the water, from his point of view, the drop will levitate.

\onslide+<1->
\begin{description} 
    \item[frame reference] Référentiel
\end{description}
\end{frame}


\subsection{Optical Illusion 2 : Stroboscopic Light !}
\begin{frame}
\frametitle{Optical Illusion 2 : Stroboscopic Light FTW !}

Main improvement : far less dangerous than the previous one.

\pause
\medskip

Use a regular dropwise and a stroboscopic light at the same frequency (quite a bit dephased).

The observer will see drops in air as if they were floating.

\pause
\medskip

This process use the POV phenomenon.
A modification of the strob phase shift makes the drops go up and down.

\medskip

\onslide+<1->
\begin{description} 
    \item[dropwise] Goutte à goutte
    \item[POV] Persistance of Vision
    \item[phase shift] Décalage de phase, déphasage
\end{description}
\end{frame}

\section{Real Levitation}
\begin{frame}
\frametitle{Real Levitation}

Core concept :

\begin{center}
Compensate the gravity field's action
\end{center}

\pause

In optical illusions, we used gravity to achieve our goals, now...

\begin{center}
Let's fight against it !
\end{center}
\end{frame}

\begin{frame}
\frametitle{Real Levitation: 3 ways}
\begin{itemize}
    \item Weighlessness
    \item Supraconductors
    \item Acoustic Levitation
\end{itemize}
\end{frame}


\subsection{Real Levitation 1 : Weightlessness}
\begin{frame}
\frametitle{Real Levitation 1 : Weightlessness}

In a spatial station, or a parabolic flight, a water drop will float in air.

\pause

In a spatial station, speed and distance make Earth's gravity field really weak.

\pause

In a parabolic flight, the aircraft acceleration is greater than the gravity constant, so materials inside are not under
its control anymore.

\onslide+<1->
\begin{description} 
    \item[weighlessness] Apesanteur
\end{description}
\end{frame}

\subsection{Real Levitation 2 : Supraconductors}
\begin{frame}
\frametitle{Real Levitation 2 : Supraconductors}

Combination of :

\begin{itemize}
    \item Extreme power of superconductor magnet
    \item Diamagnetic properties of water
\end{itemize}
\end{frame}

\begin{frame}
\frametitle{What's a Diagmagnetic material}

\begin{quotation}
Damagnetism is the property of an object or material which causes it to create a magnetic field in opposition to an
externally applied magnetic field.
\end{quotation}
\begin{flushright}Wikipedia\end{flushright}

\pause

If you apply a strong magnetic field to a water drop, the drop will create a opposite field.
\medskip


The two magnetic forces are opposite so the drop (or any object containing a lot of water) will levitate.
\end{frame}

\subsection{Real Levitation 3 : Acoustic Levitation}
\begin{frame}
\frametitle{Real Levitation 3 : Acoustic Levitation}

Acoustics describes waves through fluid, gases or solids.

\medskip

These waves affect the material in which they "travel", creating pressure and speed differences.

\medskip

Using some particular waves, it's possible to create pressure "steps".
\end{frame}

\begin{frame}
\frametitle{Real Levitation 3 : Acoustic Levitation}
The most powerful example : stationnary waves which create nodes and anti-nodes of pressure.
\medskip


Each node is like a pressure step where you can "put" objects like a water drop.

\onslide+<1->
\begin{description} 
    \item[pressure nodes] noeud de pression (zone de faible pression absolue)
    \item[pressure anti-nodes] ventre de pression (zone de forte pression absolue)
\end{description}
\end{frame}

\end{document}
