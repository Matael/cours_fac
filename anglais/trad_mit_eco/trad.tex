\documentclass[a4paper,11pt]{article}

	\usepackage[utf8]{inputenc}
	\usepackage[T1]{fontenc}
	\usepackage[french]{babel}

	\usepackage[top=2cm,right=2cm,left=2cm,bottom=2cm]{geometry}

	\usepackage{url}

	\usepackage{multicol} % texte sur plusieurs cols.
    \setlength{\columnseprule}{.4pt}

	\title{MIT Predicts That World Economy Will Collapse By 2030}
	\author{}
	\date{}

\begin{document}
\begin{multicols}{2}
	\maketitle

\section{Original}

Forty years after its initial publication, a study called The Limits to Growth is looking depressingly prescient.
Commissioned by an international think tank called the Club of Rome, the 1972 report found that if civilization
continued on its path toward increasing consumption, the global economy would collapse by 2030. Population losses would
ensue, and things would generally fall apart.

The study was — and remains — nothing if not controversial, with economists doubting its predictions and decrying the
notion of imposing limits on economic growth. Australian researcher Graham Turner has examined its assumptions in great
detail during the past several years, and apparently his latest research falls in line with the report’s predictions,
according to Smithsonian Magazine. The world is on track for disaster, the magazine says.

The study, initially completed at MIT, relied on several computer models of economic trends and estimated that if things
didn’t change much, and humans continued to consume natural resources apace, the world would run out at some point. Oil
will peak (some argue it has) before dropping down the other side of the bell curve, yet demand for food and services
would only continue to rise. Turner says real-world data from 1970 to 2000 tracks with the study’s draconian
predictions: “There is a very clear warning bell being rung here. We are not on a sustainable trajectory,” he tells
Smithsonian.

Is this impossible to fix? No, according to both Turner and the original study. If governments enact stricter policies
and technologies can be improved to reduce our environmental footprint, economic growth doesn’t have to become a market
white dwarf, marching toward inevitable implosion. But just how to do that is another thing entirely.

\columnbreak
\section{Translation}


Quarante ans après sa première publication, une étude appellée "Les Limites de la Croissance" semble désespérement
présciente.

Commanditée par un "think tank" international appellé le "Club de Rome", le rapport de 1972 avait montré que si la
civilisation continuait sur son chemin vers une consommation grandissante, l'économie globale s'éffondrerait à l'horizon
2030.

S'ensuivraient alors des pertes démographiques et tout tomberait finalement en lambeaux.

L'étude était, et reste, pour le moins controversée , certains économistes doutant de ses prédictions et décriant
l'idée d'imposer des limites à la croissance économique. Le chercheur Graham Turner a examiné les assumptions de
l'étude en profondeur au cours des dernières années et apparement, ses dernières recherches viennent confirmer les
prédiction du rapport, d'après le Smithsonian Magazine. Le monde cours au désastre nous apprend la revue.

L'étude, initialement menée par le MIT, repose sur une série de modèles informatiques des tendances économiques et elle
estimait que si les choses ne changeaient pas sérieusement et que les Humains continuaient de consommer les ressources
naturelles à cette vitesse, le monde s'épuiserait à un moment donné. le prix de l'essence va alors culminer (certains
pensent que c'est déjà le cas) avant de dévaler la pente opposée d'une courbe en cloche, alors, les besoins en
nourriture et en services continueront d'augmenter. Turner annonce que les données du monde réel entre 1970 et 2000
collent aux prédictions draconniennes de l'étude : "Aujourd'hui, la sonnette d'alarme retentit clairement. Nous suivons
une trajectoire qui n'est pas viable." annonce-t-il au Smithsonian.

Est il impossible de réparer cela ? Non, si l'on en croit à la fois Turner et l'étude originale. Si les gouvernement
actaient en faveur de politiques et technologie plus strictes, cela pourrait réduire notre empreinte environnementale.
La croissance ne doit pas devenir une naine blanche, avaçant inexorablement vers une explosion.

Mais faire simplement ça est un sujet à part entière.

\end{multicols}
\end{document}
