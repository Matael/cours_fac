\documentclass[11pt]{article}

	\usepackage[utf8]{inputenc}
	\usepackage[T1]{fontenc}
	\usepackage[english]{babel}

	\usepackage{geometry}
	\geometry{top=1cm, right=1cm, bottom=1cm, left=1cm}

	\title{Congressional Candidates and Incumbents Should Be Required To Pass Rigorous Academic and Professional Examinations Before Being Given Seat In Public Office}
	\author{Found on product-boy.com}
	\date{http://product-boy.com/2011/08/09/congressional-candidates-and-incumbents-should-be-required-to-pass-rigorous-academic-and-professional-examinations-before-being-given-seat-in-public-office/}

\begin{document}
	\fontfamily{phv}
	\selectfont
	\maketitle

Checks and Balances. It’s the supposed foundation of democratic transparency that ensures the US government will always
maintain certain levels of power allotment, disallowing any one branch disproportionate power and influence.
But what about the checks and balances required for Congress people to take office in the first place? Sure, they are
elected into their positions of public service by their voter constituencies, but should that (and age) be the end all
of requirements to be a part of the elected leadership of the United States?

It certainly isn’t easy. Getting elected into Federal Congress is one of the more difficult tasks any aspiring
politician can hope to achieve. Getting to the upper echelons of local government and party politics, in order to be
nominated for Congress takes years of dedication, sometimes starting with door to door soliciting of political support
based on a certain platform. Other than the simple age requirements, our representatives spend years establishing a
voter base, strong political dispositions, and relationships with sources for capital (campaign finance is, after all,
quite the tricky business). But what happens when they convince a large constituency that they’ve got the people’s best
interest in mind, and actually get voted into Congress? Is that enough to ensure the best and the brightest, not to
mention the most qualified people are running our government?

Universities, private organizations, and various other professions require rigorous testing of their candidates before
positions are allotted to them. Sure, most (if not all) people in Congress have been to college, some have a laundry
list of professional experience, and others, like our president, are actually lawyers from some of the most highly
esteemed institutions of higher education in the world (Obama is a Harvard Law alum, in case you were wondering).

This is not enough, especially when public politics in a capitalist republic are concerned, where there are numerous
avenues and opportunities for “moral grey areas” to take hold, and shift priorities. Our leaders need to be held to
higher standards. The caveats of political correctness will have us saying “to each their own”, to make us ignore the
disillusionment of having the likes of Palin, Bachmann, Barton, Lewis, and their ilk representing us and making sure the
best interests of the people and the Constitution are what they consistently work for. The ugly truth is, some of our
leadership gets elected simply because they’re master orators catering to the lowest common denominator of public
intelligence, and that is extremely dangerous.

If history is any guide, the recent debt ceiling debacle should be enough to demand a test of intellect before taking
the upper echelons of public office. Bi-partisan bickering and distraction, economic brinkmanship, and a complete lack
of intelligent decision making and compromise are slowly becoming the norm on Capitol Hill, and it’s about time there
were checks and balances in place to make sure these professionals remain just that, professionals.

We often hear the old adage, “Don’t complain about a problem unless you’ve got a way to fix it.” So, here is a start: A
preliminary list of subjects that every single person in Federal and State public office should be tested on, and
proficient in, before being given their seat.

\section{The Constitution Of the United States of America}

It is imperative that our leadership is not only well versed, and can recite passages from the founding document of our
nation, but they should intrinsically understand it, and all of its underlying ideals and caveats. This means questions
which go deeper than “which part of the Constitution is…” They should be questions that test the candidates on the
legally binding, philosophically important, and modern applicatory characteristics of the document. Think LSAT and BAR
exam level questioning on this one document alone, the ultimate case study.

\section{Economics and Finance}

Not just the run of the mill banker’s education, either. Our leadership should be fully aware of the underlying
ambitions and history of the financial institutions that have such a huge grip on our national well-being. Capitalism
should be allowed to thrive, but we can’t consistently have cries for a more “open/free market” from the corporate
world, while they continue to receive taxpayer welfare any time they’re in trouble. A free market means if you make
mistakes, you die. The cyclical market bailouts and continued disproportionate support of the banking ecosystem by our
leadership is a recipe for disaster. So when an industry analyst or master economist from a large banking institution
sits in front of our simple leaders and prophesizes complete disaster unless a taxpayer bailout is issued…that’s just
stupid, and extremely dangerous for the populous, as has been proven in recent years.

\section{History and Sociology}

These two go hand in hand, as the study of history will often describe in accurate detail the overall makeup of a
society at any given time in any given nation. Socio-economic modality, the dynamics of multi-cultural constituencies,
and the effects of mass psychology should be studied and mastered. Furthermore, for the sake of foreign policy, our
leaders should have a thorough understanding of geo-political context, and cultural relativism when dealing with allies,
enemies, and neutral nations alike.

\section{Science and Engineering}

The idea here isn’t to have every Congressperson become an engineer or research scientist. Rather, the core ambition of
this type of testing would be to educate potential political leaders about the importance of innovation, the direct
positive correlation it has towards economic growth, and the exponential betterment of the quality of life. By
understanding FULLY the ancillary benefits of a robust and continuously growing and well-supported scientific community,
our leadership will understand the importance of maintaining and incentivizing real scientific discourse, rewarding
substantial and applicable discoveries, not hyper-complex mathematical models which game markets and create massive
wealth with almost zero labor.

This list is far from exhaustive, and the questions of logistics still remain. Who would compile the exam? Who would
administer it? How often would it change to guard against cheating, etc.?

It isn’t a complete fix, but the idea sure is a start, because it isn’t too much to ask that the people running the US
are smart enough to understand, at a basic level, the entire breadth of issues that might come across their desks.
\end{document}
