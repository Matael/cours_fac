\documentclass[a4paper, 11pt]{report}
    \usepackage[utf8]{inputenc}
    \usepackage[T1]{fontenc}
    \usepackage[french]{babel}

    \title{{\Huge Rapport}\\Projet Personnel de l'Etudiant\\
        Salles de Spectacle}
    \author{Mathieu Gaborit — Jean-Suliac Defontaine\\Professeur : Christophe Ayrault}
    \date{Université du Maine | Mai 2012} 

\begin{document}
    \fontfamily{phv}
    \selectfont

    \maketitle
    
    % table of contents
    \tableofcontents
    \newpage


% Section : Introduction
\section*{Introduction}

Au cours de cette première année, il nous a été demandé de préparer une présentation de notre projet professionel (ou du moins d'un projet possible).
En définitive, il s'agissait d'une recherche concernant un métier et d'une prise de contact avec un professionnel.

Le Projet Personnel de L'Etudiant (PPE) présenté ici est principalement une recherche autour de la conception des salles de spectacle.

Le choix a été fait de croiser plusieurs sources d'informations, de se baser sur nos connaissances, et sur une interview formelle (d'autre personnes ont été contactées avant ou pendant).

L'objectif était, plutôt que de travailler que sur un point de vue, d'en confronter plusieurs.

L'un de nous a déjà évolué dans le monde du spectacle côté technique, l'autre, musicien, connait aussi la scène.
Nous avions chacun une affinité plus ou moins marquée avec cet univers et nous avons cherché à connaître les points de vue de personnes impliquées dans la création de salles.

Le choix de se baser sur notre connaissance du monde du spectacle plutot que sur des recherches documentaires est introduit par le manque cruel d'information et le peu de résultats de nos recherches.

Même si nous savions déjà ce dont nous allons parler, traiter ces connaissances dans ce cadre en les confrontant avec celles de notre binôme constitue tout de même une découverte.
Ici, nous nous sommes évertués à aggréger ce que nous savions en un ensemble cohérent, nous avons ainsi découvert de nouveaux liens entre les informations et pu répondre à la problématique.

\section{Au commencement}

Avant toute chose, il a été convenu de recenser les informations déjà en notre possession et de définir des objectifs.

Les informations {\em fiables} à proprement parler n'étaient pas forcément très nombreuses mais certains de nos a priori se sont vite confirmés.

\subsection{Avant : le point de vue du technicien}

Nous nous sommes concertés quand à l'implication des techniciens (futurs utilisateurs) dans la conception de la salle.

Après quelques prestations, il apparait que les salles sont plus souvent faite pour être {\em belles} que pour être {\em pratiques}.
Ce postulat est tiré directement de nos expériences ainsi que de quelques discussions antérieures avec des professionnels du monde du spectacle (on pourra notament citer SMHE, entreprise de prestations évènementielles).

Afin de vérifier cette proposition, une interview de Lilian {\sc Duffrechoux} (ITEMM) a été réalisée.

\subsection{Avant : le point de vue de l'acousticien}

Vis à vis de l'acousticien, nous pensions qu'il était {\em systématiquement} consulté avant toute construction.
Consulté ne signifie pas forcément écouté...

Nous manquions cruellement d'information de ce côté là, Mr {\sc Duffrechoux}.

\subsection{Avant : le point de vue de l'architecte}

Maître de la conception d'un batiment, le point de vue de l'architecte semble primordial.
Avant d'aller plus loin, nous pensions que les architectes concevant les salles de spectacle étaient spécialisés, informés au moins.

Il n'était pas absurde de penser qu'une formation à l'acoustique des salles leur était dispensée.

Une brève entrevue a eu lieu.

\subsection{Avant : L'artiste}

La salle est, le temps d'un spectacle, un écrin pour une oeuvre.

Bien que l'un des principaux intéressés, nous pensions les artistes relativement peu contactés.

Un spectacle peu souvent s'adapter à une salle, l'inverse n'est vrai que dans une bien maigre mesure.


\section{Résultats}

\subsection{Interview de Lilian {\sc Duffrechoux}}

Naissance du projet :

\begin{itemize}
\item sollicitation de musiciens, d'associations (utilisateurs de la salle pour discuter des caractéristiques spécifiques du lieu) ainsi que d'une force politique pour avis et évaluation du budget.
\item Concours entre cabinets d'architectes puis choix des mandataires
\item Construction de la salle.
\end{itemize}

Lors de la construction :

\begin{itemize}
\item Partie artistique d'architecture : façade, ...
\item Côté pratique des installations \& acoustique de la salle : partie mise a l'écart
\end{itemize}

Personnes en collaboration sur le chantier :

\begin{itemize}
\item Architectes \& bureaux d'études associés spécialisés dans l'acoustique (facultatif)
\item Scénographe (espace scénique, hauteur de plafond, ...)
\item Techniciens du son :  peuvent effectuer des demandes spécifiques et techniques dans le but de rendre la salle pratique en prévision du matériel électronique à déployer (avis consultatifs pas forcément pris en compte)
\item Musiciens en fin de projet : estimé le rendu sonore de la salle
\item Acousticiens : pas forcément contacté. Ils sont malheureusement contactés à la fin du chantier pour apporter des corrections après coup.  Ils effectuent donc des aménagements et des corrections plutôt que participer à la création.
\end{itemize}

Cabinet d'architecte spécialisé dans l'acoustique des salles de spectacle : non mais repérage d'après les réalisations du cabinet  (absence de spécialisation).

Selon Mr {\sc Duffrechoux}, le 1er avis donné devrait venir d'une collaboration entre acousticiens et scénographes ; s'en suivrait un travail d'architectes basé sur les restrictions, les contraintes et les demandes des spécialistes de l'acoustique et de la scène.
Le problème de l'architecte vient de son orientation trop poussée sur l'aspect extérieur du bâtiment, sur la décoration, l'emballage de la salle.

L'architecte est apte à bâtir la salle, mais seul les spécialistes du son peuvent donner les indications nécessaires a l'élaboration d'un lieu à l'acoustique en accord avec son rôle de salle de spectacle.

Malheureusement, le coût d'une intervention d'ingénieurs acousticiens et de scénographes est souvent trop élevé du point de vue des investisseurs.


L'Espace de Vie Étudiante, par exemple, a lors de sa construction vu intervenir les ingénieurs acousticiens du laboratoire de recherche de l'université du Maine, dont les recommandations n'ont pas été prise en compte.

\subsection{Discussion avec un architecte DPLG}

Un architecte a accepté de répondre à quelques questions.

Les informations tirées de cet échange sont les suivantes : 

\begin{itemize}
    \item l'architecte ne se voit pas dispenser une vraie formation à l'acoustique, tout au plus une introduction sommaire
    \item les études acoustiques sont réalisées par un bureau d'étude mandaté pour cela
\end{itemize}


\section{Discussion}

Les recherches menées et l'interview en elle même mettent en évidence des failles dans notre conception de ce groupe de métiers.

M. {\sc Duffrechoux} nous a clairement signifié que les techniciens n'étaient que rarement consultés durant la conception du salle de spectacle.
L'exemple du Grand Atelier (Espace de Vie Etudiante de l'Université du Maine) a été pris notament pour montrer que  l'ITEMM n'avait pas été consulté.

Un précédent échange avec un membre du département acoustique de la faculté des sciences avait déjà mis en évidence le non respect des recommendations fournies.

Enfin, lors d'interventions au Grand Atelier, des étudiants de l'ENSIM (faisant alors des tests) avaient confirmé la piètre qualité de la salle en dépit des personnes compétentes présentes sur place.

\medskip

Un certain nombre de personnes appartenant au monde du spectacle nous ont eux aussi confirmé qu'ils n'avaient été consultés qu'a posteriori, généralement lors de la phase d'équipement de la salle.

La formation des architectes comporte une introduction sommaire à l'acoustique, celle ci pouvant être approfondie au cours d'une spécialisation.
A noter que cette spécialisation n'est pas obligatoire pour la construction de salles de spectacle.

Quand un avis d'expert est nécessaire (et si le client le souhaite), un bureau d'étude est contacté.
Les prestations sont alors relativement coûteuses ce qui reffroidit parfois les clients.


Selon M. {\sc Duffrechoux}, un bureau d'architecture ne comprend pas forcément (et même rarement) d'acousticien et fait donc appel à un bureau d'étude spécialisé le cas échéant (ou pas).
D'un point de vue purement naïf, on pourrait envisager une étroite collaboration entre architectes et acousticiens au sein d'un même bureau spécialisé dans ce genre de conceptions.
La place de personnes du monde du spectacle dans ce bureau serait un véritable atout et, mieux encore : si architectes et acousticiens connaissaient le milieu pour lequel ils travaillent, cela produirait des batiments vraiment adaptés.

\section{Enfin}

A la lumière de ces informations, il nous a fallu revoir notre conception des liens entre ces métiers.

On constate que, la plupart du temps, la seule chose qui les lie vraiment est la salle en elle-même, sans son aspect pratique.

A titre personnel, ce qui m'a semblé le plus abérant est la non-consultation courante d'un bureau d'étude acoustique au cours de la conception.
Bien sûr, et on le remarque particulièrement lors de l'utilisation de la salle, il en résulte une acoustique parfois médiocre et mal optimisée.
Un second effet est parfois plus gênant : le monde du spectacle court en permanence contre la montre, une mauvaise acoustique de base occasionne des pertes de temps parfois monstrueuses.

Un aspect qui n'a été que peu développé ici est le point de vue du mandataire lui-même.
En effet le {\emph "client"} n'a pas (du moins pas toujours) un budget extensible, et l'on doit donc prendre en compte cette contrainte : on préfèrera souvent quelque chose de beau à voir que de beau à entendre.

Reste à examiner un problème récurrent dans les cas plus modestes : les demandeurs souhaitent parfois (souvent ?) réutiliser un batiment existant pour base.
Cela évite en effet une bonne partie d'un gros-oeuvre souvent couteux.

Reste quand même qu'avec toute la bonne volonté du monde, les miracles n'existent pas : on ne transforme pas une ancienne conserverie de poisson en salle de spectacle (Saint Gilles Croix de Vie, Vendée) ou bien une gare en théatre (Fontenay le Comte), etc...


Ces obstacles purement pécuniers sont autant de freins à la réalisation de salles à la fois belles et agréables (tant à l'oreille que pour y travailler).

\end{document}
