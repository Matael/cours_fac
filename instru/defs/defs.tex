\documentclass[a4paper, 11pt]{article}
    \usepackage[utf8]{inputenc}
    \usepackage[T1]{fontenc}
    \usepackage[french]{babel}
    \usepackage{geometry}
    \geometry{top=3cm, right=2cm, bottom=2cm, left=2cm}

    \title{Instrumentation : Définitions}
    \author{L1 SPI}
    \date{Avril 2012}

\begin{document}
\maketitle

\begin{description}
    \item[Mesure] Ensemble d'opérations ayant pour but la détermination de la valeur d'une grandeur
    \item[Grandeur] Attribut d'un corps ou d'une substance qui est susceptible d'être distingué qualitativement et déterminé quantitativement
    \item[Mesurage] Action de mesurer
    \item[Mesurande] Grandeur à mesurer
    \item[Unité] etalons pour la mesure de grandeurs physiques
    \item[Mesure directe] Comparaison avec un étalon sans passer par le mesurage d'autres grandeurs
    \item[Mesure indirecte] La valeur du mesurande est obtenue après mesurage d'autres mesurandes puis en appliquant une relation liant les différents mesurandes appellée loi de comportement
    \item[Sensibilité] Paramètre exprimant la variation du signal de sortie en fonction de la variation du signal d'entrée
    \item[Dynamique] L'étendue de mesure est le domaine de variation possible du mesurande. Elle est définie par les valeurs minimales et maximales. La dynamique est le rapport, exprimé en dB, entre la valeur maximale et la valeur minimale.
    \item[Bande Passante] La bande passante est la bande fréquentielle dans laquelle le capteur peut être utilisé
    \item[Directivité] La directivité traduit les directions privilégiées dans lesquelles le transducteur capte de l'énergie
    \item[Capteur actif] Fournit en sortie de l'énergie électrique sous forme d'une tension ou d'un courant
    \item[Capteur passif] Ne fournit pas d'énergie mais l'un de ses paramètres varie sous l'action du mesurande (ex: thermorésistance)
    \item[Couplage] Lors d'un mesurage, le mesurande cède de l'énergie au capteur : c'est le couplage. Le couplage entre mesurande et capteur est nécessaire pour effectuer un mesurage. Notons que la présence du capteur modifie le mesurande.
    \item[Numérisation] Etape consistant en la transformation d'un signal analogique vers un signal numérique. Elle se déroule en 2 temps : l'échantillonnage (découpage temporel et régulier du signal analogique) et la quantification (évaluation de l'amplitude de chacun des échantillons par rapport à des références et une échelle déterminée)
    \item[Générateur] élément permettant de fournir à l'actionneur une consigne donnée
    \item[Amplificateur] amplifie la consigne avant d'attaquer l'actionneur et adapte son impédance de sortie pour transmettre un maximum d'énergie
    \item[Actionneur] élément permettant de transmettre de l'énergie au système. En contact direct avec le système.
\end{description}
\end{document}
